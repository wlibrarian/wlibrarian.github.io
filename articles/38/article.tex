This issue's lecture reviews were written by Isky Mathews, Ms.~Stone and
the Editor.

\subsection{Middle East Society - The Great War and its consequences}

\textbf{Professor Ali Ansari, Director of the Institute for Iranian Studies at
	the University of St Andrew's}



	\subsubsection{Historical context}\label{historic-context}

	Before World War I, there were two main imperial bodies in the region
	known as the Middle East: Persia and the Ottoman Empire. World War I led
	to Ottoman breakup. This was an especially important formative period
	for the Middle East. One should always be sceptical of coverage of the
	Middle East: Sykes-Picot, ever-present, never saw their plan fully
	implemented - nor did anyone else. Russia's importance in the region has
	often been underestimated; it has historically had significant interests
	in the Middle East.

	\subsubsection{The Iraq War}\label{the-iraq-war}

	2001 and 2003 disturbed this post-Great War settlement. The Americans
	attempted and somewhat managed to reshape the region in their own image.
	A particularly salient feature is the innovation of attempting to spread
	democracy. Ideally, this would allow a quick turnover; it turned out to
	lead to a significant loss of stability.

	\subsubsection{Obama}\label{obama}

	Obama's legacy is important. All Presidents attempt to make up for their
	predecessor's failings. In this case, post-Bush exhaustion led to a
	perhaps premature withdrawal from the region. The second main feature of
	his rule was a (very) partial attempt to move a little closer to the
	pre-1979 situation where Iran was one of the main American partners in
	the region. A constructive relationship required a partner; this came
	not in the form of Ahmajinedad but Rouhani. The most significant step
	forward was nuclear. To some degree, this relationship is not based on
	material concerns but was emotional. It was hoped, especially by the
	Americans, that initial nuclear progress would lead to a broader
	rapprochement. This fits in, perhaps, to belief in Whig history, that
	is, history whose moral curve is long but bends towards justice. The
	pivotal moment may have been 2011/2012, where sanctions began to cripple
	the Iranian economy.

	\subsubsection{Modes of interaction}\label{modes-of-interaction}

	There are two possible modes of interaction. The first involves the use
	of a thin end of the wedge, with the eventual aim of slotting the fat
	end of the wedge through too. The second involves having a strong base
	of initial progress upon which to build later, in a stable fashion. The
	former appears to have been employed under Obama.

	Iranian thought values businessmen - Trump fit this description, so the
	Iranians thought they could make a deal. They were, however, not
	completely correct.

	To Trump, the deal's provisions are flawed in one crucial way - they are
	limited to a decade. The assumption that after ten years Iran will be
	sufficiently sound and integrated into the world economy is at best not
	entirely guaranteed. Another limitation of the agreement is that it has
	not been legally enshrined - it was pushed through by Presidential
	diktat, not legislative consent; just as the then-President gave, so too
	the current President may take away.

	\subsubsection{Who to trust?}\label{who-to-trust}

	Professor Ansari concluded with an anecdote. An analyst, once, noticed
	that posters of President Obama had been plastered all over Iranian
	cities. He failed to notice that Obama was pictured next to Shemr - an
	ancient villain in Shi'ism.

	\subsubsection{Questions}\label{questions}

	\begin{enumerate}
		\def\labelenumi{\arabic{enumi}.}
		\item
		How best can the Joint Comprehensive Plan of Action (JCPOA) be
		supported? The JCPOA makes political development contingent on
		economic development. A lack of Western investment in crucial
		industries has blunted its benefits to Iran. Crucially, access to
		dollar markets is lacking in Iran, curtailing investment and growth.
		These two concessions would lead to economic development and goodwill.
		\item
		What is the difference between the two methods of interaction
		previously described? Why must a first step necessarily be unstable?
		The Iranians decided to negotiate exclusively on nuclear issues. All
		further discussion was precluded. Had a grand bargain been struck,
		other issues may have been included; now that nuclear-related and
		sanctions-related issues have been somewhat solved, the Americans have
		fewer bargaining chips for later negotiations or negotiations on the
		opening of negotiations.
	\end{enumerate}



\subsection{Stipatores Honorati}\label{stipatores-honorati}



	After a quick change of venue from the Refectory to Purcell's common
	room (the tables and chairs had been cleared out of the Refectory ready
	for redecoration over Exeat), cake was served, tea was made and we all
	settled down to hear William speak on the history of apple cake. Norfolk
	apple buns had been set aside in favour of Dorset apple cake (the buns
	do not keep well and are also best made with almonds, which are
	obviously a banned substance) but nobody voiced any complaints and the
	quick disappearance of everyone's cake spoke well of the quality - your
	correspondent, for one, thought it excellent.

	William then led us on a quick whirl through various apple cakes from
	Dorset and Somerset, a West Country poem and the revelation that Norfolk
	apple buns are actually a surprisingly recent invention created by Pye
	Baker of Norfolk, finishing off with a reading about apples. Questions
	were then welcomed, including why do we so often see apple paired with
	cinnamon (no conclusion was drawn - perhaps a topic for further
	research) and what everyone's favourite variety of apple was. Many
	different suggestions were made, with high scorers being Braeburns and
	Cox's, but Asian apples had a late surge and one dissenting voice
	plumped for oranges instead.

	A spirited debate then sprung up about the differences between muffins
	and cupcakes. Some members argued that they were the same thing, but
	others highlighted several differences including, but not limited to,
	the fact that the muffin escapes its case and the cupcake stays
	contained. The meeting then drew to a close with Colin the Caterpillar
	being accepted as the November cake, despite not being homemade.



\subsection{Biology bites}\label{biology-bites}



	This talk at Huxley Society was a rounding up of the science
	department's ``Biology Week'', consisting of multiple successive
	mini-presentations on different aspects of biological study.

	The talk began with Hein Mante's presentation on sharks. He firstly gave
	great detail on their evolutionary history:

	\begin{itemize}
		\item
		400 million years ago, the first cartilaginous fish have been observed
		to have evolved.
		\item
		At 350 million years ago, shark scales were first formed in various
		pre-shark species.
		\textbf{Placoid scales are, in fact, one of the distinctive features of sharks. They produce small-scale vortices around the sharks’ bodies which improve hydrodynamic efficiency and appear to have evolved from vertebrate teeth, as they contain a central pulp cavity supplied with blood vessels.}
		\item
		320 million years ago, the first shark, \textit{Cladselache}, was
		recognised.
		\textbf{It should be noted that this makes sharks one of the oldest and most successful branch of fish.}
		\item
		Around 100 million years ago, a great explosion in shark species
		occurred for somewhat unknown reasons.
		\item
		65-60 million years ago, the Great White and Megalodon sharks become
		distinct species.
		\item
		1 million years ago, the last Megalodon specimen dies.
	\end{itemize}

	He then continued on to discuss their predatory behaviour and whether
	their ``man-eating'' reputation in public opinion is well-deserved. Hein
	mentioned that just 5 deaths per annum are caused by shark attacks, in
	comparison to Malaria's 780,000/annum and AIDS' 1,800,000/annum
	\textbf{(or the 20 deaths per year caused by improper use of vending machines)}.
	He observed that shark size affects predatory behaviour and that 2.3m
	appears to be a ``critical length'' beyond which sharks become
	significantly more aggressive, attempt to eat much larger prey than
	would normally be possible to swallow and also exhibit a much greater
	chance of trying to eat indigestible objects, such as pieces of coral,
	sponge or human waste.

	Hein ended by considering sharks' main predator, orcas. It was mentioned
	that they flip sharks upside-down using their tails - it appears this
	action causes a shark's brain to flood with endorphins and as a result
	they become docile and motionless, even with the orca biting into them!

	Mr.~Moore's presentation followed, beginning with an anecdote in which
	he had overheard some Upper School Westminsters declaiming Biology and
	those who took it along some remarks suggesting its inferiority to the
	other sciences. He jokingly expressed his unhappiness at this comment
	and said that Biology is often portrayed as a ``Mickey Mouse'' subject;
	he humorously concluded from this that the way to understand this remark
	is to examine the history of Mickey Mouse himself.

	He noted how Mickey Mouse's appearance and personality has changed
	significantly. No longer does Mickey Mouse smoke or steal or commit any
	sort of wrong-doing and he has gotten a larger cranium, eyes and
	exaggerated limbs. Mr.~Moore explained that humans naturally find these
	features cute and pleasurable to look at - in fact, these features are
	found in nearly all baby animals and this is part of the reason most
	creatures have an urge to look after their young. Konrad Lorenz, the
	Nobel Prize winning biologist, was mentioned for his work in describing
	and categorising these precise physiological differences and follow up
	work by neuroscientist Stephen Hamann showed that the common responses
	to these features were involuntary amygdala responses, so nearly all are
	lying if they claim to not find such features attractive.

	He described the phenomenon known as \textit{neoteny}, where adults of a
	species retain features normally only found in juvenile specimens. He
	then pointed out that humans appear to have developed a form of neoteny,
	since adults of \textit{Homo Sapiens} share many facial characteristics
	with young primates.
	\textbf{It is believed that this was evolutionarily advantageous because having these attractive features probably made it more likely for a given "neotenic" individual to be helped by others.}

	Mr.~Moore concluded that there is Biology in Mickey Mouse but the
	opposite is not true.

	Brandon Tang then began his short presentation on cannibalism in nature
	by showing that it appears \textbf{(perhaps oddly)} often throughout the
	animal kingdom: pray mantis who eat their mates, tadpoles eating their
	siblings, hippos etc. He detailed a notable example in which a primate
	mother kept the body of her dead baby with her for at least a month,
	before proceeding to eat increasingly large portions of the body. He
	elaborated that this was not uncommon with primates; pack leaders have
	been seen to be eaten and families of primates have been seen to travel
	as a group eating other primates they encounter.

	The traditional explanation for this was that in certain crowded areas
	of a forest, there may be great sexual competition and competition for
	food and that this leads in certain situations to primates killing
	others of their own kind, either out of some form of stress or
	necessitation.

	Brandon said that he felt these explanations were inadequate as it
	doesn't quite explain, for example, why direct relatives display a
	tendency to not attack each other but rather group together nor does it
	give any explanation for the situation of the mother carrying the dead
	baby as mentioned above. He gave a counter proposition: that this
	instead suggests such primates have a much higher level of complexity to
	their social behaviour than previously thought. He finished with the
	conjecture that perhaps their emotional intelligence has been
	underestimated and that they are quite capable of complex feelings
	towards others.

	The final presentation was a talk on sloths and their peculiar habits by
	James Ferner. It was mentioned that in Costa Rica, there are two species
	of sloths: the three-toed and two-toed sloths. Along with the obvious
	difference in the number of claws they have on their hands, the two
	species have numerous physiological differences - this observation along
	with DNA samples have led to the hypothesis that the two species come
	from quite distinct branches of the tree of life.
	\textbf{The two-toed sloth is believed to have been tree climbing for some million years but the three-toed sloth may have originated from the Giant Ground Sloth and become tree dwelling much more recently.}
	If this was found to be true, James notes, then these sloths species
	would demonstrate one of the most striking examples of convergent
	evolution ever discovered.

	Both species are folivores (leaf-eaters) with 4 stomachs and are
	extremely slow-moving. They come down once a month to go to the loo and
	this is when they are most vulnerable - they are predated on by snakes,
	harpy eagles and cougars. James then mentioned that their hair is so
	thick and curly that it is often seen green from the algae cultures that
	grow in it. They have developed, it seems, a form of symbiotic
	relationship with these plants, since the curls of the hair appear to
	exist to trap moisture which promotes algae growth, where the algae's
	growth provides a form of camouflage in the foliage to sloths. This
	appears to make it more difficult for harpy eagles to find them in the
	canopy.

	James ended the series of talks by mentioning their mating rituals,
	since he believed them to be quite humorous. It appears that females
	emit a high-pitch shriek from their tree to make it clear that they are
	ready to mate and then male sloths become much more active than normal,
	with some two-toed sloths seen swimming across lakes to get to the
	female. If multiple males arrive at roughly the same time, they can
	fight \textbf{(quite slowly!)} for dominance in mating.
