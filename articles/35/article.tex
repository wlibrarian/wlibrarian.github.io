\textbf{Some of this article has been published before; now that it has been completed, both parts of the article have been put together.}

\subsection{Introductory}


	About 100 thousand Hong Kong residents marked June 4 with an annual commemoration of the Tiananmen Square massacre. These commemorations provide an important reminder of the true nature of the Chinese state. They are important on their own, by virtue of the sheer barbarity involved in the massacre. Yet they now are perhaps important for us in a self-interested calculus.

	China matters. Almost everyone acknowledges now that it matters, and that whatever rhetoric surrounds its rise is not wholly ``hype'', as it might have been of Japan. It matters not only in the UK, where $7\%$ of our electricity supply will soon be reliant on Chinese goodwill and engineering, or in those parts of the Eurasian landmass which are to be connected by its One Belt One Road initiative, but also in the entire world with universal and undeniable force, when the vast majority of the computing equipment essential to the working of the modern world is manufactured in China.

	When such a rise occurs, it is important that the nature of the rising power is analysed, both out of an intrinsic need to know of one's fate as interaction with China changes both in intensity and in balance and out of a need to make informed choices as to those dimensions in interaction with China.

	First, three initial examples will help to illustrate the scope and aim of this article.

	Yang Shuping, a student at the University of Maryland, said in a graduation speech that she felt the "fresh air of free speech" during her time at university Apart from some embellishment in relation to air pollution in her city of Kunming, it's not clear exactly what was taken issue with.

	``We need to open up and embrace all the suggestions from outside world," claims the first speaker. "I would be so piss[ed] off if anyone described my country with destruction\footnote{The audio isn't entirely clear.}," she continues.

	A few days before, China held a ``Belt and Road initiative" meeting in Beijing. One notable absence was that of Lee Hsien Loong, Prime Minister of Singapore; Singaporean media speculated that this was due to an incident several months prior, wherein a Terrex armed vehicle was impounded by Hong Kong authorities while returning from a training exercise in Taiwan.

	Several months earlier, a mainland legal expert proposed an adjustment in the ratio of foreign judges in Hong Kong, after a foreign judge, David Dufton, sentenced police officers who assaulted protestors in the 2014 Occupy Hong Kong protests to jail.

	Although all these incidents appear to be unrelated, they are a symptom of a deeper malaise which touches the heart of every state and establishment, but does so with particularly problematic consequences in states which have experienced recent political ruptures or lack conventional accountability mechanisms. China, which suffers acutely from both problems, is run by a leadership both in great need of and with few sources of legitimacy.

	These travails may seem to be the unimportant rumblings of a faraway Eastern nation. They are not. They will soon, if they do not already, affect you and me as much as they affected Yang and Dufton. When China was introverted, it was easy enough for its political oppression to stay within China; now that it has opened up, it is willing and able to exert pressure on outside actors to change their behaviour, often for the worse. It is therefore important to question what motivates the Chinese state and how it comes to decisions, in order to determine what is likely to occur as a result of this externally as well as internally.

	This article will explore the ways in which the Chinese state can achieve legitimacy. It concludes that, in the absence of democratic reform, the Chinese state will either collapse or become increasingly stridently nationalist, associating Chinese identity with so-called ``Chinese values'' of dictatorship and international dominance, which presents a number of problems for other states, both in their treatment of Chinese citizens and in their relations with China.


\subsection{The foundations of the Chinese nation state}


	On 12 February 1912, millennia of tradition in the world’s longest continuously surviving civilisation came to a halt, as the child Emperor abdicated. An interregnum marked by instability, military domination by warlords and misery ceased upon communist victory on 1 October. Outwardly, the revolution took the form of a comprehensive movement away from ideals which dated back to the preceding millennium. This was, too, the case inwardly. Between 750,000 and 1,500,000 were killed in a campaign of mass and popular depravity.

	The democratic aspirations of the Chinese people may have been illusory, limited merely to Western-connected intellectuals. Yet 12 February marked also a time in which the Chinese people threw off the shackles of imperial authority. The speed of change is striking. Nominally one of the world’s most powerful people in 1912, Puyi, the child-emperor of China in the last days of empire, spent a decade in Communist prison camps before retiring to write a pro-Mao memoir-cum-apologia, in which he wrote ``I covered up my crimes in order to protect myself”, in reference to his justifications for cooperation with the Japanese in Manchuria.

	Inwardly, however, one pressure remained - the need for legitimacy. The removal of millennia of institutional inertia presented the greatest opportunity for democratic renewal that China has ever faced. Although democratic systems occasionally suffer from legitimacy problems due to low turnout, inherent in democratically elected governments is at least some degree of popular support, and so there is a floor of legitimacy. Dictatorships' floor of legitimacy is such that they only are overthrown when a sufficient proportion of the people feel so incensed by régime incompetence or manevolence that they attempt to overthrow the government.

	First, we must ask ourselves where legitimacy is derived from, particularly in a Chinese context.

	China is not entirely alien to the rest of the world. Its people care about life, liberty (in a sort of collective sense, although perhaps not individually) and other basic needs. In this sense, Chinese régimes need, like other régimes, to ensure that people living under them are adequately fed, housed and clothed. The failure of the republican leadership to do this during the civil war can be said to have lead to their downfall, because it encouraged support for the communists. This need, however, is not absolute. Many Chinese dynasties limped on while their people languished in poverty, and the communists themselves. In that sense, legitimacy derived from living standards exists in China as it exists universally.

	There are also ideological concerns and ways of achieving legitimacy. For thousands of years, the idea of the Mandate of Heaven created loyalty to the emperor, at its basic heart. However, it was sufficiently flexible to allow new emperors or dynasties to take over, when ``Heaven'' decided that the emperor or a dynasty had become too venal and corrupt to deserve its mandate. In this way, it provided a useful tool by which to, somewhat external to material concerns, rule. As a result of its great and increasing age, it managed to survive for thousands of years as the basis of Chinese government; we conclude from this that it was quite successful in its aim.

	The Mandate of Heaven is not the only way by which ideological legitimacy can be created. The communists attempted to replace this with a utopian vision of a Maoist society. The degree to which this was successful is quite difficult to ascertain. On the one hand, there was great enthusiasm in the campaign against ``intellectuals'' and other undesirables during the Cultural Revolution and Great Leap Forward. On the other hand, those same people who probably were around persecuting unfortunates quickly reversed their stance and displayed great flexibility in adapting to first ``Socialism with Chinese characteristics'' and then its true face, ie. some form of state capitalism, with all the rent-making opportunities that implied. Were all these hundreds of millions of people really genuinely convinced that Mao was a great visionary who had all the answers, but simultaneously able to adjust? Or, as is more likely, were they moulded by circumstance, perhaps to some degree motivated by genuine fervour, but really largely simulated by each other? One shout to ``kill 1000 intellectuals'' would likely have been followed by another hundred in a sort of collective inaction problem - what if one is the next to be deemed a reactionary for insufficient revolutionary fervour?

	How does this apply to the new régime? First, any new state will have problems in finding popular legitimacy. This is inevitable when revolution necessarily causes all the old institutions of government to, at the very least, lose power or fragment. As a result, many basic services which citizens rely on the government to provide and/or would expect of any government, including at the very least security, are likely not to be provided. This problem was compounded by a deliberate policy of repression of ``intellectuals'', ie. people who had a university education or vaguely had knowledge of how to run things which China needed, like power stations or transportation networks. Given this, the foundation of the Chinese state was not built on material concerns - indeed, it had to be actively shifted away from such concerns because initial administration was so poor.

	Second, the Chinese state was deprived of the idea of the Mandate of Heaven. Not only were they not monarchical, which meant that such ideas were already largely impossible, given that the Mandate of Heaven was given to an emperor and his/her descendants, collectively comprising a dynasty, they were also diametrically opposed to the intellectual fabric which gave credence to such ideas, in that they sought to destroy traditionalist ideas of God, religion and emperor. We can therefore discount divine legitimacy as a source of inspiration.

	Third, what is left is pure ideology. The sole mechanism by which the communists were able to gain legitimacy was ideological struggle, a mechanism which is hard to sustain in this solitary manner. Not only would it have, at some point, come into conflict with the necessity of preventing state collapse - legitimacy is a mechanism to stay in power, but if external factors mean that even with perceived legitimacy one collapses, it does not help, a reliance on ideological struggle fails because of its inherent idealism, which is incompatible with practical problems causing a failure to achieve utopia.

	Precipitately, the foundations upon which the Chinese state were built were entirely precarious, and especially ill-suited to China.


\subsection{Normalisation}


	This situation could not go on, and it did not. Many will be familiar with at least an outline of what happened under Deng Xiaoping. Economic liberalisation, and a decision to cease the continual shooting-oneself-in-the-foot policies which marked Maoism, meant that the state could refocus itself on achieving the first type of legitimacy, ie. that based on material concerns.

	This worked, for a while. China lifted 800 million people out of poverty and is now the largest economy in purchasing power terms in the world. Yet 10\% growth is barely sustainable. China has been growing at this rate for nearly four decades. It cannot continue for much longer - growth is now down to below 7\%, and may fall further. Independent sources of information which might tell us without government interference what the real economic situation is in China, but until then we are dependent on the government for such information.

	The Chinese leadership, however, recognises that it is very difficult to continue to use ideological fervour to sustain the legitimacy it has created based on economic growth. This puts it in a very dangerous situation, because the measure needed to ensure long term growth and hence long term stability are in conflict with measures which would simulate the economy in the short term in order to stave off loss of legitimacy.

	It is difficult to determine whether this is true. Government approval ratings in China are often above 90\%, but even if they are conducted impartially, it is possible and indeed quite likely in a nation which went through the trauma of the Cultural Revolution that these polls are affected by culturally ingrained reminders to toe the party line, both because it is correct in an objective sense, and because such toeing is in one's self-interest.

	More easily, one can analyse the measures that the Chinese government is taking to ensure that it retains control. This, as a first step, indicates that it is worried about its control of the country.


\subsection{Increasing dictatorship}


	Liberalisation, and democratic stagnation/reversal, in China, largely took place on three tracks: first, internal party mechanisms of control, second, information control, and third, other civil liberties like freedom of movement and freedom from torture.

	After the excesses of Mao, the communists in China recognised that they too would have to face up to the ``bad emperor problem'': with a high level of state control, one can enrich oneself and the country with good policymaking, but should a bad ``emperor'' or ``paramount leader'' as the case may be come to power, the whole of China would face severe difficulties. Mao was their ``bad emperor'', and he taught the communists the virtue of true collective leadership. Although the communists have not become paragons of liberal or democratic virtue, it is notable that the infighting which characterised the Party under Mao significantly abated under the mechanism of collective leadership. The solution of collective leadership works by slowing down the acquisition and exercise of power by its diffusion. In doing so, one has fewer ``bad emperors'', even though progress is slowed. It is indubitably superior to the days of Mao or Stalin, and is superior in that it prevents massive shifts in Party power from one faction to another in the hope of achieving total poewr - instead one vies for positions within the Party and the individual powers they offer.

	As for information, after Mao, it was significantly freed up. This reached its peak under Zhao Ziyang, but still remained better throughout all of China than under Mao. Chinese media report on a number of social and environmental issues, including smog, contamination from industrial pollutants, and local government incompetence. Although this coverage is often scaled back or retrospectively deleted after censors become nervous, the Chinese people still remember articles after they are deleted. Similarly, somewhat liberal academics were, to some degree, free to debate, within their universities and their dormitories, with less government interference, compared with Mao-era detention and expulsion to the countryside of all academics. In mnay cases, this lead to a degree of introspection which was not present in Mao-era China. This provided great potential for a gradual and optimised transition to democracy and accountability, for it would not have occurred in a vacuum of knowledge.

	It cannot be pretended that China was at any point a state which respected certain basic rights of its citizens, like their bodily autonomy or property rights. At no point did the culture of state impunity which pervaded instituions like the ``People's Armed Police'' abate significantly - they were still free to oppress the people. However, what did change was that they were used less. There was less state violence post-Mao. Fewer properties were seized. Torture was used less, and in some cases the judiciary even admitted that it was wrong. This was real progress - certainly not enough for those who still are victimised by it, but still real.

	The 18th National Congress of the Communist Party of China marked the start of a turning point in this slow movement towards democratic legitimacy. It initially appeared to be the start of faster movement. Corruption trials of senior party officials finally went ahead. Chinese officials didn't stop mourners from flocking to the house of Zhao Ziyang, a reformist, to mourn his death. Xi was a new face, representing a new era.

	These illusions soon wore off. The corruption trials were really a tool to achieve greater power within the Communist party for Xi and his friends. Indeed, Xi himself is said to be suspiciously wealthy, and he and his family hold a number of properties outside China. Chinese officials probably didn't mind letting people mourn Zhao Ziyang because very few people remember him and they felt that their hold on power was greater than ever before, with their control of the main means of transmission of information, ie. the internet. When people posted pro-Zhao messages on internet message boards, they were deleted, but their posters were not rounded up and detained.

	Perhaps a new face, Xi is most certainly not doing anything other than shoring up the dictatorial powers of the party.

	Xi has been named the ``Core of the Party'', and many commentators suggest that he has more power than any leader since Deng Xiaoping. Some have even suggested that he may wish to continue after his customary pair of five year terms in positions in the Central Military Commission and Party, though he may resign as President as is constitutionally mandated. In his attempts to corral the paty into submission, it is feared that he may have alienated so many officials that the only way to ensure that he is not arrested just as he has arrested some former officials is to stay in power, and that means that we must wait until his death, not his retirement, to see the back of him.

	Information controls have been re-imposed. Liberal academics find themselves under increasing pressures to make universities conform as a tool for communism. No longer are the authorities content to leave them as bastions of ineffective and bourgeois liberalism, never seriously mounting a challenge to Communist leadership - they want them to actively participate in their programme of brainwashing the people.

	Repression continues unabated. Dissidents are denied access to their lawyers. An environmental documentary which observers initially thought would be permitted under a new censorship policy of tolerating environmental polemics was soon prohibited. Lawyers of dissidents are occasionally detained. Sometimes their lawyers are detained. The state appoints its own lawyers for them, who are often incompetent and fail to defend them at all - the reason why should be obvious.

	This is already bad. It indicates that the government of the most populous government in the world does not even pay lip service to ideals of human dignity and accountability, and that anyone who even suggests it ought to runs the risk of finding their lives turned upside down. Actual observance of these norms is out of the question. But it is also normal. Lots of governments have acted like this, and their peoples have still managed to recover to some degree. Many dictatorships have also ruled over large numbers of people. Part I is a reminder that a powerful nation is increasingly at odds with the values which not only the West but much of the world holds dear. Part II of this series will explore the ways in which China is spreading its influence across the world, and why China therefore presents a problem perhaps unprecedented in the history of humanity in its propagation of its own values, if they even exist as such and are not simply base interests.

	The broad thesis of the first of this series of articles is that China
	is unlikely to liberalise in the near future, and that the most likely
	path is one of gradually increasing oppression. This article will
	explain why China is increasingly able to implement such goals, both at
	home and abroad.

	A broad effect of modernity is that everything has become bigger in
	magnitude: people, numbers of people, armies, weapons, speeds, data and
	coercive capacity have all massively increased in scale, complexity and
	efficacy. Hegemony in the tenth century was local, and necessarily so.
	Hegemony with modernity affects nearly everyone.

	The events predicted in this article are likely to affect vast numbers
	of people outside China. They will not necessarily happen instantly, and
	there may be reversals within the broad trends described. Nevertheless
	the author has a relatively high degree of confidence in the general
	direction described.


\subsection{Domestic control}\label{domestic-control}

\subsubsection{Propaganda}\label{propaganda}


	Control of information may not change individual axiology but it
	certainly changes individuals' decisions as to how best to implement
	that axiology. We should not deny the former, for many of our
	instrumental beliefs, and indeed core axiological beliefs, are imparted
	by the outside world, and so are influenced by those who control access
	to information. Nevertheless, the primary concern of the modern day
	Chinese censor is not to change axiological beliefs in the value of
	family, riches or success. Rather, it is to change the instrumental
	values which are perceived to be the best way to achieve them. Xinhua's
	press release on Xi Jinping thought claims that such thought is an
	attempt to become "a great modern socialist country that is prosperous,
	strong, democratic, culturally advanced, harmonious, and beautiful by
	the middle of the century."\footnote{http://news.xinhuanet.com/english/2017-10/24/c\_136702802.htm}

	Propaganda's value is derived from its ability to change thought.
	Previously, the Chinese censor was constrained by a need to retain some
	credibility. The efficacy of Chinese propaganda was necessarily limited
	by the tradeoff between message and credibility.

	China, however, has moved on from real life. The rise of a peculiarly
	Chinese internet has been the primary cause of what has happened. It is
	unique in two ways: first, the quality and quantity of its adoption, and
	second, the degree of state control over it.


\subsubsection{Information control}\label{information-control}


	China has the world's most internet users - 732 million\footnote{http://www.scmp.com/tech/china-tech/article/2064396/chinas-internet-users-grew-2016-size-ukraines-population-731-million}.
	That China's total population is 1.4 billion masks something important -
	almost all young people use the internet. The quality of China's
	internet usage reflects two important characteristics of this growth.
	First, its magnitude incentivises internet development in a way which
	smaller markets elsewhere do not. Second, that most of this growth was
	recent meant that the way that the internet is used in China is
	primarily mobile-led and significantly more adapted to the latest
	technologies - epayments and so on - than other markets.

	Chinese internet usage is therefore different to that in the West. It
	includes much more than its Western equivalents - payment systems for
	not only online shopping but also physical shopping, a greater embrace
	of the ``gig'' economy in providing platforms for freelancers to sell
	their services, and heavy integration between and within apps are
	particularly salient features of one representative and popular app -
	WeChat\footnote{https://www.economist.com/news/business/21703428-chinas-wechat-shows-way-social-medias-future-wechats-world}.
	\textit{The Economist} claims that the unique characteristics of the
	Chinese market - its size, a propensity to possess multiple devices and
	the cultural phenomenon of red packets - were part of this shift.
	Another important factor is, however, that the Chinese internet is
	deeply separated from the rest of the world. Although one can reach the
	outside world from the inside, and outsiders can reach the Chinese
	internet, such connexions are often slow - a number of senior Chinese
	political figures have complained about this, as have
	scientists\footnote{https://www.hongkongfp.com/2017/03/19/chinese-scientists-speak-great-firewall/}.
	The connexion between China and the rest of the world is akin to having
	a dirt path between two major cities - though there may be few
	insurmountable obstacles, most would not attempt to use it. Users are a
	little like drivers, in that they will quickly abandon that which is
	inconvenient\footnote{https://www.thinkwithgoogle.com/data-gallery/detail/mobile-site-abandonment-three-second-load/}.
	That internet users are fickle means that they are very likely to use
	Chinese tools instead - especially since Western tools now do not appear
	to offer any particular benefit over their Chinese equivalents.

	Chinese control over the internet broadly sues the same technologies
	which are available to Western countries - filtering, vast numbers of
	administrators and moderators, and other heuristics-based algorithms
	which block terms. Two things differ: first, political will, and second,
	the overtness with which the Chinese government controls the internet.
	In the West, perhaps the most draconian régime is that of the United
	Kingdom, where § 58 of the Terrorism Act (2000) states:

	\begin{enumerate}
		\item
		A person commits an offence if---

		\begin{enumerate}
			\item
			he collects or makes a record of information of a kind likely to be
			useful to a person committing or preparing an act of terrorism, or
			\item
			he possesses a document or record containing information of that
			kind.
		\end{enumerate}
		\item
		In this section ``record'' includes a photographic or electronic
		record.
		\item
		It is a defence for a person charged with an offence under this
		section to prove that he had a reasonable excuse for his action or
		possession.
		\item
		A person guilty of an offence under this section shall be liable---

		\begin{enumerate}
			\item
			on conviction on indictment, to imprisonment for a term not
			exceeding 10 years, to a fine or to both, or
			\item
			on summary conviction, to imprisonment for a term not exceeding six
			months, to a fine not exceeding the statutory maximum or to both.
		\end{enumerate}
	\end{enumerate}

	We should note, of course, that a number of potentially ``reasonable''
	excuses have been rejected by the courts, including attempting to
	understand terrorist ideology\footnote{https://www.theguardian.com/world/2012/dec/06/woman-jailed-al-qaida-material-on-phone}.

	In China, control goes much further. Not only is the scale of censorship
	greater, but that there is censorship is censored. Not only are
	``terrorist'' materials proscribed - those advocating democracy or
	attacking the régime in any way whatsoever are prohibited\footnote{https://www.economist.com/news/special-report/21574631-chinese-screening-online-material-abroad-becoming-ever-more-sophisticated}.

	Willingness to politicise the internet - to use its technical
	capabilities in a political fashion as opposed to using its technical
	capabilities apolitically to facilitate political actions, combined with
	its pervasive influence over Chinese lives, distinguishes China from other nations.

	Any claim that China will liberalise because its people demand it must
	overcome this new reality of information control in China. The Communist
	Party are now here to stay - they have always had a desire to, and now
	they have the power to.


\subsection{International control}\label{international-control}


	China is increasingly able to assert its power abroad. This occurs in
	two ways. First, China increasingly has a strategic advantage which
	enables it to assert its power over other nations. Second, Chinese
	people, in occupying positions elsewhere, are often vulnerable to
	mainland influence in a number of ways - cultural, political, familial
	and economic.


\subsubsection{China's strategic
	advantage}\label{chinas-strategic-advantage}


	China has, broadly, five advantages, primarily over the United States,
	but also over the rest of the world.

	First, China has the world's largest manufacturing base\footnote{https://data.worldbank.org/indicator/NV.IND.MANF.CD?locations=CN-US-JP}.
	Its advantage here is especially pronounced in the field of electronics.
	Informatic systems are important in all fields of life - most Western
	economies rely on them to function. No other country is able to match
	China in quantity, and, increasingly, quality. China will be able to
	exploit this to its advantage as it increasingly becomes less dependent
	on other nations for development, as exports as a proportion of its GDP
	have decreased\footnote{https://data.worldbank.org/indicator/NE.EXP.GNFS.ZS?locations=CN}
	and are likely to decrease further.

	Second, China controls the natural resources critical for such
	manufacture. China supplies 85\% of the world's rare earths\footnote{https://thediplomat.com/2017/08/revisiting-rare-earths-the-ongoing-efforts-to-challenge-chinas-monopoly/}.
	These are crucial in building informatic systems.

	Third, China has a population advantage, especially when compared to
	other potential competitors. Russia and the United States have
	significantly smaller populations at a global level. At a local level,
	only India can match its population; all other nations which China will
	seek to influence in the coming years - Japan, South Korea, and those in
	South East Asia, have far smaller populations.

	Fourth, China has a capacity for coördinated action which few other
	nations possess. It is relatively trivial to see that American
	dysfunction is unlikely to disappear in the near future. India, too,
	despite BJP political hegemony, retains its everlasting capacity to slow
	reforms down. This gives it an advantage over other nations in adopting
	new technology, upgrading its military, increasing economic growth with
	infrastructure projects, maintaining domestic political control and
	increasing domestic production for strategic purposes.

	Fifth, China increasingly leads the world technologically. Consortia of
	Chinese scientists have been particularly important in recent
	developments in quantum physics\footnote{http://www.bbc.co.uk/news/science-environment-40294795}.
	They have also been at the forefront of genetic research\footnote{http://www.nature.com/news/chinese-scientists-to-pioneer-first-human-crispr-trial-1.20302}.


\subsubsection{People}\label{people}


	There are 50 million Chinese abroad\footnote{http://www.asiapacific.ca/sites/default/files/filefield/researchreportv7.pdf}.
	Many of these Chinese occupy important positions in academia, politics,
	law and so on. They are particularly prominent in Singapore and South
	East Asia as a whole, where their traditional prosperity has often made
	them targets for political oppression, as occurred in Indonesia under
	Suharto.

	Most Chinese have relatives in China. China has no compulsion against
	attacking relatives\footnote{http://www.telegraph.co.uk/news/2016/03/28/another-chinese-dissident-says-relatives-detained-over-letter-cr/}\textsuperscript{,}
	\footnote{http://www.reuters.com/investigates/special-report/china-uighur/}.
	Given this, it appears likely that China will start to, if it has not
	already done so, use its overseas Chinese in dubious ways.

	This presents an important moral problem. Many Chinese live abroad. It
	would normally be abhorrent to impose a blanket ban on the employment of
	Chinese in sensitive positions. Nevertheless, when the assumption that
	would justify such a ban - that Chinese people cannot be trusted - is
	undermined (in this case, by the Chinese government), it is unclear
	which should be prioritised. We should expect to increasingly be faced
	with such problems.

	Chinese people in public life who are not spies will also face problems.
	There will be a permanent cloud of suspicion over their heads.

	We shall not pretend here to provide a solution. It suffices to say that
	Chinese infiltration could undermine the very basis of Western liberal
	democracy, but so too any attempt to respond. Ultimately, immigration
	policy in Western nations may become significantly more xenophobic as
	a result of these developments. Other nations may also attempt to exploit
	their diasporas too, causing restrictions on their nations.

	Those who reside in dictatorial countries will find that their choices
	are greatly reduced. Unable to flee, they will find themselves trapped in
	their own increasingly sustainably dictatorial countries.


\subsubsection{A collective action
	problem}\label{a-collective-action-problem}


	It is clear that China has a vast number of carrots to offer other
	nations in attempting to impose its will. There is, of course, a classic
	collective action problem here. Those who refuse to coöperate on a
	principled basis lose, but the Chinese people gain nothing from such
	refusals, for China will always find someone else to interact with.

	Coördinated opposition could come from an alliance of relatively large nations in the Pacific region. However, this would require political unity internal and external. The United States, presently distracted by other matters and seemingly unable to confront the threat which China poses, appears unlikely to resolve its domestic political dysfunction in the near future. Hence such an alliance is unlikely to form. Attempts to form a four-party alliance have repeatedly failed due to Indian concerns over Chinese interference in Australia.


\subsubsection{Winds of change}\label{winds-of-change}

	We underestimate the power that a sense that ``progress'' is one way or
	another has. During the end of history, we saw significant
	liberalisation and democratisation. This correlated significantly with
	two phenomena: first, the fall of the Soviet Union and hangers-on, and
	second, the victory of the idea of liberal democracy.

	As the Soviet Union fell, dictatorships around the world loosened
	controls or fell. In Russia, there was, for a period, genuine freedom of
	expression, and, to some degree, liberty. There is evidently none now.
	Empirically, there is a correlation.

	Why? All dictatorships need a raison d'être, even if merely for show.
	This is why dictatorships which have complete control or near complete
	control over their societies - the DPRK, China and Russia, for example -
	still require some explanation for their actions. No dictatorship has
	yet openly declared itself a kleptocracy. This is for two reasons.

	First, officials, when overly kleptocratic, cause state collapse, by
	overly extracting rent. This can be solved with an ideological basis -
	it ensures lip service, allows enforcement mechanisms to be
	significantly more effective, and motivates officials in a way which
	overt kleptocracy does not.

	Second, there is an innate human desire to justify one's actions. It is
	possible that humans are incompatible with overt kleptocracy. Even some
	of the most evil people that the world has ever seen - Hitler, Stalin
	and Mao, attempted to frame their actions as conducive to a certain
	(often warped) perception of ``the good''.

	It is most certainly not for want of material control that this occurs.

	China's dominance means that the winds of progress are perceived to blow
	towards it. In the West, confidence in democracy has declined\footnote{https://www.weforum.org/agenda/2016/12/charts-that-show-young-people-losing-faith-in-democracy/}.
	Outside it, dictators who previously claimed that the path to democracy
	was simply to be had later\footnote{https://www.thetimes.co.uk/article/iraqis-not-ready-for-democracy-says-blairs-envoy-qkf8rm6j3wf}.
	Now, it has grown - dictators no longer need to pay lip service to it,
	and they now feel free to entirely reject, down to an axiological level,
	democracy.

	This is important because even though lip service does not always result
	in significant gains in freedom for the majority of the population, the
	groups which it allows to exist are important in any post-dictatorship
	scenario. That, for example, there was continuity between colonial and
	postcolonial institutions in many nations appears to have contributed to
	stability in a number of British colonies. Where there was no continuity
	or this continuity was rejected, instability reigned. We may, for
	example, compare the Democratic Republic of the Congo and Namibia. This
	is not exactly a fair comparison for a number of reasons, but
	empirically nevertheless there is something to be said for continuity.

	When dictators feel free to destroy civil society, there is nothing
	left. Societies may genuinely be better off with dictators than under
	anarchy. Those who would resist do not.

	Resistance to oppression involves another collective action problem. If
	every single North Korean not part of the régime were to attempt to
	overthrow it, the régime would soon fail. Yet this does not happen. For
	every individual, the price of resistance is very great, and the likely
	gain very limited. Once the collective action problem is overcome,
	whether by chance or particularly egregious oppression, there is no more
	problem. Civil society organisations are crucial in overcoming
	collective action problems. They allow resistance to régimes to
	crystallise, providing a focal point for others to join and a core
	organisation.

	China's ascendency empowers dictators to copy its model, even if it does
	not take an active interest.


\subsection{Conclusion}\label{conclusion}


	China's present leadership has the tools to stay in power for a very
	long time, both internally, with its control of information, and
	externally, with its strategic advantages. As demonstrated in the first
	part of this article, it is not a particularly pleasant one. Its model
	of dictatorship will be exported, not primarily deliberately but simply
	by its changing of the international political climate. As it does,
	other nations will also be able to acquire the tools to entrench their
	leaders just as China has done.
